% % % % % % % % % % % % % % % % % % % % % % % % % % % % % % % % % % % % % % % % 
% LaTeX4EI Template for Cheat Sheets                                Version 1.0
%					
% Authors: Emanuel Regnath, Martin Zellner
% Contact: info@latex4ei.de
% Encode: UTF-8, tabwidth = 4, newline = LF	
% % % % % % % % % % % % % % % % % % % % % % % % % % % % % % % % % % % % % % % % 


% ======================================================================
% Document Settings
% ======================================================================

% possible options: color/nocolor, english/german, threecolumn
% defaults: color, english
\documentclass[german]{latex4ei/latex4ei_sheet}
\usepackage{stix}
\usepackage[ngerman]{babel} 
% set document information
\title{Leistungselektronik \\ Cheat Sheet}
\author{Raoul Duke}
\myemail{0x4723@gmail.com}
\mywebsite{www.github.com/doppelplus/CheatSheets}

% ======================================================================
% Define MathOperator
% ======================================================================

\DeclareMathOperator{\arcsec}{arcsec}
\DeclareMathOperator{\arccot}{arccot}
\DeclareMathOperator{\arccsc}{arccsc}
\DeclareMathOperator{\arcsinh}{asinh}
\DeclareMathOperator{\arccoth}{acoth}
\DeclareMathOperator{\arctanh}{atanh}
\DeclareMathOperator{\arccosh}{acosh}

% ======================================================================
% Begin
% ======================================================================
\begin{document}


% Title
% ----------------------------------------------------------------------
\maketitle   % requires ./img/Logo.pdf

% Tipps und Tricks
% ----------------------------------------------------------------------
\section{Allgemeines}
	\begin{sectionbox}
		\begin{symbolbox}{Allgemeines}
			\item Tastverhältnis: $D = \frac{\tau_i}{T}$
			\item Funktion einer Sinusspannung: $u(t) = \hat{U_s} \cdot \sin(\omega t)$
		\end{symbolbox}
		\begin{bluebox}{Physikalische Größen}
			\item $U_0$: Gleichspannung
			\item $\hat{u}$: Scheitelwert
			\item $u(t)$: zeitabhängige Spannung
			\item $T$: Periodendauer
			\item $t_i$: Impulszeit
			\item $\overline{U}$: Arithmetischer Mittelwert
		\end{bluebox}
	\end{sectionbox}
\section{Mathematische Verfahren}
	\subsection{Mittel- \& Effektivwert}
	\begin{sectionbox}
		\begin{symbolbox}
			\item Arith. Mittelwert einer Mischspannung: $\overline{u}_{di} = U_{di} = \frac{1}{T}\int\limits_0^T u_d(t)\,dt$
			\item Effektivwert: $U_{RMS} = \sqrt{\frac{1}{T}\cdot \int\limits_{\alpha}^\beta u^2_d(t)\,dt}= \frac{u_d \cdot \sqrt{2(\sin(2\alpha)-2\alpha - \sin(2\beta)+2\beta)}}{4\cdot\sqrt{\pi}}$
		\end{symbolbox}
		\begin{cookbox}{Effektivwert einer diskreten Spannung}
			\item Spannung in Spannungen mit gleichem $\hat{U}$ aufteilen.
			\item Effektivwerte der Einzelspannungen berechnen:\\ $U_{xRMS} = \sqrt{D} \hat{U}$.
			\item Quadratische Summe aller $U_{xRMS}$ berechnen: $\\ U_{RMS} = \sqrt{U_{xRMs}^2+U_{x+1RMs}^2\dots}$
		\end{cookbox}
	\subsection{Welligkeit, Klirr und Formfaktor}
		\begin{symbolbox}{Welligkeit (Ripple)}
			\item $w_U = \frac{U_{RMS}}{U_d} = \sqrt{\frac{U_{RMS\,ges}^2}{U_d^2}-1}$
			\item $w_I = \frac{I_{RMS}}{I_d} = \sqrt{\frac{I_{RMS\,ges}^2}{I_d^2}-1}$
			\item Welligkeit reiner Gleichgrößen: $w = 0$.
			\item Welligkeit reiner Wechselgrößen: $w =$ sehr groß.
		\end{symbolbox}
		\begin{bluebox}{Klirrfaktor (THD)}
			\item $K_U =   \frac{U_{RMS\, OS}}{U_{RMS}}$
			\item $K_I =   \frac{I_{RMS\, OS}}{I_{RMS}}$
		\end{bluebox}

		\begin{symbolbox}{Formfaktor}
			\item $F = \frac{U_{d\,RMS}}{U_{di}}$
		\end{symbolbox}
	\subsection{Eigenschaften netzgeführter Stromrichterschaltungen}
		\begin{tabular}{|l|c|c|c|c|c|}
			\hline
			& \textbf{M1} & \textbf{M2}& \textbf{M3}& \textbf{B2C}& \textbf{B6C}\\
			\hline
			$U_{di0}/U_S$ & 0,45 & 0,9 & 1,17 & 1,8 & 2,34\\
			\hline
			Steuergesetz  & \multicolumn{4}{c}{$U_{di0}\cos \alpha$}&\\
			\hline
			Welligkeit $w_{u0}$ &1,21 & 0,48 & 0,183 & 0,48 & 0,042\\
			\hline			
		\end{tabular}
	\end{sectionbox}
\section{Leistungsberechnung}
	\begin{sectionbox}
		\subsection{Leistungsarten}
			\begin{bluebox}
				\item \includegraphics[width=180px]{img/Leistungsarten.png}
				\item $S = U_{0\,RMS} \cdot I_{O\,RMS}$
				\item Für rein sinusförmige Verläufe gilt:
				\item $\lambda = \frac{P}{S} = \cos \phi$
				\item $S = \sqrt{P^2+Q^2}$
				\item $Q = \sin(\phi)$
			\end{bluebox}
		\subsection{Betriebsquadranten}
			\begin{symbolbox}
				\item \includegraphics[width=180px]{Betriebsquadranten.png}
			\end{symbolbox}
	\end{sectionbox}
\section{Wärmemanagement}
	\begin{sectionbox}
		\subsection{Verlustleistung}
			\begin{bluebox}
				\item Thermische Energie: Q
				\item Momentanleistung am PN Übergang: $p_v = u \cdot i$
				\item $Q = \int\limits_0^t p(t)\,dt$
				\item \includegraphics[width=180px]{TempErsatzschaltbild.png}
			\end{bluebox}
			\begin{tabular}{|l|c|c|}
				\textbf{Bauelement}	& \textbf{Kennbuchstabe} & \textbf{Temperatur}\\
				\hline
				Siliziumkristall - Junction & J & $\vartheta_J$\\
				Gehäuse - case & C & $\vartheta_C$\\
				Kühlkörper - heatsink & K & $\vartheta_K$\\
				Kühlmedien - ambient & U / A & $\vartheta_A$
				
			\end{tabular}
	\end{sectionbox}
\section{Mittelpunktschaltungen}
	\begin{sectionbox}
		\subsection{Nomenklatur}
			\begin{bluebox}
				\item $i_d\,u_d$: Zeitverläufe von Strom und Spannung
				\item $I_d\,U_d$: In den Zeitverläufen von $i_d$ und $u_d$ enthaltene Mittelwerte
				\item $u_T$: Zeitlicher Verlauf der Spannung an einem Thyristor
				\item $u_S$: Zeitlicher Verlauf der Netzspannung
				\item $U_S$: Effektivwert der Netzspannung
				\item $U_N$: Effektivwert der verketteten Spannung
				\item $d$: Ausgangsgröße
				\item $T$: Transistor
				\item $S$: Strang
				\item $N$: verkettet Größe
			\end{bluebox}
		\subsection{Welligkeit}
			\begin{bluebox}
				\item $w_U = \sqrt{\frac{U_{RMS}^2}{U_d^2}-1}$
			\end{bluebox}
		\subsection{Einphasige Mittelpunktschaltung M1}
			\subsubsection{Aufbau und Funktion}
				\includegraphics[width=110px]{Mittelpunktschaltung.png}
			\subsubsection{Steuergesetz}
					Rein ohmsche Last: $U_{di\alpha} = \frac{\hat{U}_S}{2\pi}\cdot (1+\cos  \alpha)$

					$\frac{U_{di\alpha}}{U_{di0}} = \frac{1+\cos \alpha}{2}$

		\subsection{Zweiphasige Mittelpunktschaltung M2C}
			\includegraphics[width=110px]{ZweiphasigeMittelpunktschaltung.png}

			$u_{s12} = u_{s1}-u_{s2} = u_N \cdot \frac{N_2}{N_1}$
			\subsubsection{Stromglättung}
				Bei induktiver Last gilt: $u_d = u_R + u_L = i_d\cdot R+L\cdot \frac{di_d}{dt}$
			\subsubsection{Steuergesetz}
				Bei nicht lückendem Betrieb ergibt sich für $U_{di\alpha}$:

				$U_{di\alpha} = \frac{1}{2\pi}\int\limits_\alpha^{2\pi+\alpha}u_d(\omega t)d(\omega  t) = \frac{2\cdot \sqrt{2}}{\pi}\cdot \hat{U}_S \cdot \cos \alpha$
		\subsection{Dreiphasige Mittelpunktschaltung M3C}
			\includegraphics[width=120px]{M3C.png}

			$U_{RMS} = \hat{U}_S \sqrt{\left[ \frac{1}{2}+\frac{3}{4\pi}\cdot \frac{\sqrt{3}}{2}\right]} = 0,8405\cdot \hat{U}_S$
			\subsubsection{Steuergesetz}
			$U_{di\alpha} = \frac{3\cdot \sqrt{3}\cdot \hat{U_s}}{2\pi}\cdot \cos \alpha = \frac{3\cdot \sqrt{3}\cdot \sqrt{2}\cdot U_{s}}{2\pi} \cdot \cos \alpha$

	\end{sectionbox}
\section{Gleichstromsteller im Einquadrantenbetrieb}
	\begin{sectionbox}
		Prinzipieller Aufbau Gleichstromsteller

		\includegraphics[width=120px]{DCSteller.png}
		\subsection{Tiefsetzsteller}
			\begin{bluebox}
				\item \includegraphics[width=160px]{Tiefsetzsteller.png}
				\item $u_{SZ}(t) = \frac{\hat{U}_{SZ}}{T_S}\cdot t = U_{Steuer}$
				\item $\frac{\hat{U}_{SZ}}{T_S}\cdot t_{ein} = U_{Steuer}$
				\item $t_{ein} = \frac{U_{Steuer}}{\hat{U}_{SZ}}\cdot T_S$
				\item \includegraphics[width=160px]{ZeitverlaufTiefsetzsteller.png}
				\item Tastgrad: $D = \frac{t_{Ein}}{T_S}$
				\item Schaltbedingung:
				\item $u_{Komp} > 0 \Rightarrow $ MOSFET eingeschaltet $u_0(t) = U_d$
				\item $u_{Komp} < 0 \Rightarrow $ MOSFET ausgeschaltet $ u_0(t) = 0$
				\item Mittelwert der Ausgangsspannung: $U_0 = \frac{t_{ein}}{T_S}\cdot U_d = D \cdot U_d$
				\item $T_S = \frac{1}{f_S}$
				\item $t_{Ein} = \frac{U_{Steuer}}{\hat{U}_{SZ}}\cdot T_S$
				\item Resonanzfrequenz: $f_C = \frac{1}{2\pi \sqrt{L\cdot C}}$
				\item L und C sind so zu wählen: $f_C/f_S = 0,01\Rightarrow \frac{1}{2\pi\sqrt{L\cdot C}} = 0,01\cdot f_S$
				\item Stromwelligkeit: $\Delta_{iL} = \frac{u_L}{L}\cdot t_{ein} = \frac{U_d-U_0}{L}\cdot t_{ein}$
			\end{bluebox}
			\subsubsection{Lückender Betrieb}
				\begin{bluebox}
					\item $I_{Lg} = \frac{1}{2}\cdot i_{L\,peak} = \frac{t_{ein}}{2L}\cdot (U_d-U_0) = \frac{D\cdot T_S}{2L}\cdot (U_d-U_0) = I_{0g}$
					\item $\frac{U_0}{U_D} = \frac{D^2}{D^2+\frac{1}{4}\cdot \frac{I_0}{I_{L\,gmax}}}$
					\qquad $D = \frac{U_0}{U_d}\cdot \sqrt{\frac{\frac{I_0}{I_{L\,gmax}}}{1-\frac{U_0}{U_d}}}$
				\end{bluebox}	
	\end{sectionbox}
	\begin{sectionbox}
		\subsection{Hochsetzsteller}
			\begin{bluebox}
				\item \includegraphics[width=160px]{Hochsetzsteller.png}
				\item Der Mittelwert der Ausgangsspannung $U_0$ ist höher als der Mittelwert der Eingangsspannung $U_d$.
				\item $U_d = U_L = L \cdot \frac{di_L}{dt}$
				\item $i_L = \int U_d\,dt = \frac{U_d}{L}\cdot t = \frac{(U_d-U_0)}{L}\cdot t$
				\item \includegraphics[width=180px]{ZeitverlaufHochsetzsteller.png}
				\item $\frac{U_0}{U_d} = \frac{T_S}{T_S-t_{ein}} = \frac{1}{1-D}$
			\end{bluebox}
			\subsubsection{Lückender Betrieb}
				\begin{bluebox}
					\item $I_{Lg} = \frac{1}{2}\cdot i_{L\,peak} = \frac{t_{ein}}{2L}\cdot U_d = \frac{D}{2L}\cdot T_S\cdot U_d = \frac{T_S}{2L}\cdot D \cdot U_0 \cdot (1-D)^2$
					\item 
				\end{bluebox}
	\end{sectionbox}
\section{Gleichstromsteller im Zweiquadrantenbetrieb}
	\begin{sectionbox}
		\subsection{Zweiquadrantensteller mit Stromumkehr}
			\includegraphics[width=180px]{ZweiquadrantenstellerStromumkehr.png}
		\subsection{Zweiquadrantensteller mit Spannungsumkehr}
			\includegraphics[width=180px]{ZweiquadrantenstellerSpannungsumkehr.png}
			\subsubsection{Steuergesetz}
				\begin{bluebox}
					\item Nicht lückender Betrieb: $\frac{U_0}{U_d} = 2 \cdot D_{TA+} -1$
					\item Versetzte Taktung: $\frac{U_0}{U_d} = (D -1)$
				\end{bluebox}


	\end{sectionbox}
\section{Gleichstromsteller im Vierquadrantenbetrieb}
	\begin{sectionbox}
		\subsection{Grundlagen}
			\includegraphics[width=180px]{DCStellerVierQuadranten.png}
			\begin{bluebox}
				\item Die Verriegelungszeit bezeichnet das Zeitintervall, in dem beide Schalter einer Halbbrücke gleichzeitig abgeschaltet sind.
				\item $u_0(t) = u_{AN}(t)-u_{BN}(t)$
			\end{bluebox}
		\subsection{Pulsbreitenmodulation mit zwei Spannungsniveaus}
			\begin{bluebox}{Mittelwerte $U_{AN}$ und $U_{BN}$}
				\item $ U_{AN} = \frac{U_d \cdot t_{ein}+0\cdot t_{aus}}{T_S} = U_d \frac{t_{ein}}{T_S} = U_d \cdot D_{TA+}$
				\item $ U_{BN} = \frac{U_d \cdot t_{ein}+0\cdot t_{aus}}{T_S} = U_d \frac{t_{ein}}{T_S} = U_d \cdot D_{TB+}$
			\end{bluebox}
			\begin{bluebox}{Schaltbedingungen}
				\item $T_{A+}, T_{B-}$ ein wenn: $u_{Steuer} > u_\Delta$
				\item $T_{A-}, T_{B+}$ ein wenn: $u_{Steuer} \leq u_\Delta$
				\item \includegraphics[width=180px]{ZeitverlaufVierquadrantensteller.png}
				\item $u_\Delta = \hat{U}_\Delta \cdot \frac{t}{T_S/4}$ mit $-\frac{T_S}{4} < t < \frac{T_S}{4}$
				\item $t_1 = \frac{u_{Steuer}}{\hat{U}_\Delta}\cdot \frac{T_S}{4}$
				\item $D_{TA+} = \frac{t_{ein}}{T_S} = \frac{2\cdot t_1 +\frac{T_S}{2}}{T_S} = 2\cdot\frac{t_1}{T_S}+\frac{1}{2} = \frac{1}{2}\left(1+\frac{u_{Steuer}}{\hat{U}_\Delta}\right)$
				\item $T_{TB+} = 1-D_{TA+}$
				\item $U_0 = U_{AN}-U_{BN} = U_d \cdot D_{TA+}-U_d\cdot D_{TB+} = U_d\cdot \frac{U_{Steeuer}}{\hat{U}_\Delta}$
			\end{bluebox}
	\end{sectionbox}
	\begin{sectionbox}
		\subsection{Pulsbreitenmodulation mit drei Spannungsniveaus (PWM3)}
		\begin{bluebox}{Schaltbedingungen}
			\item $T_{A+}$ ein, wenn $u_{Steuer} \geq u_\Delta$, $T_{A-}$ ein, wenn $u_{Steuer} < u_\Delta$
			\item $T_{B+}$ ein, wenn $-u_{Steuer} \geq u_\Delta$, $T_{B}$ ein, wenn $-u_{Steuer} < u_\Delta$
			\item $D_{TB+} = \frac{\frac{T_S}{2}-2t_1}{T_S} = \frac{1}{2}-\frac{2t_1}{T_S}$
			\item \includegraphics[width=180px]{PWM3Verlauf.png}
		\end{bluebox}	
	\end{sectionbox}
\section{Umrichter}


% ======================================================================
% End
% ======================================================================
\end{document}